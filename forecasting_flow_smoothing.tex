\subsection{Flow Smoothing}
\label{sec:smoothing}

%Smoothing the flows based on the global and local trends
%After we extract a set of unified vectors to summarize the original flows, we face two major issues to construct the vector field:
Sparse and noisy flow data often generate non-smooth flow patterns that cannot be used for accurate prediction. In order to mitigate the effect caused by the data sparseness and noise, we propose a new flow smoothing method based on local and global trend estimation. The rationale behind our algorithm is that individuals in a crowd tend to follow dominant paths of the crowd~\cite{Kok:2015:Crowd}. 
%In flow analysis, if the data has a high degree of sparse and noise, it will cause broken and non-smooth flow patterns.
%Also, it is also highly likely that inaccurate predicted results will be suggested.

%The key idea behind our method is that individuals in a crowd tend to follow dominant paths of the crowd~\cite{Kok:2015:Crowd}.
In our algorithms, for each sub-space, we adjust directional density with consideration of neighbor sub-spaces' trends and a global movement that considers the entire space. 
%Figure~\ref{fig:smoothing} illustrates the main idea of our smoothing approach.
%For this, we define local neighbor sub-spaces of $sp_i$  
% $N(sp_i) = \left\{sp_j \in \Omega \ |\ sp_j\ is\ adjacent\ to\ sp_i\right\}$. 
First, local neighbor sub-spaces of $sp_i$ is defined by $N(sp_i) = \left\{sp_j \in \Omega \ |\ sp_j\ is\ adjacent\ to\ sp_i\right\}$ and  the local directional trend of $sp_i$ is computed using the neighbor representative vectors $\mathcal{R}_{sp_j}$, $sp_j \in N(sp_i)$. %Figure~\ref{fig:smoothing} (A) shows the neighbors (gray) of the center sub-space (white).

Next, we define the global movement of a sub-space as a major movement of a larger space including the sub-space.
%As described in the previous section, each sub-space will have a set of representative vectors $\mathcal{R}_{sp_i}$ representing its directional density. 
In order to compute a global trend, we extract the major movements from the trajectories of the entire space using a density-based trajectory clustering algorithm~\cite{Lee:2007:Trajectory}.
Figure~\ref{fig:smoothing} (A) shows an example computation process. Here, the representative vectors $\mathcal{R}_{centerSpace}$ of the center sub-space are almost evenly distributed in multiple directions before smoothing but the major local and global trends move toward the bottom-right. Thus, after smoothing based on the trends, the vector $V_4 \in \mathcal{R}_{centerSpace}$ heading toward the bottom-right becomes a major vector.

The next question is that which sub-spaces should be affected by the global trend since there is uncertainty in the influence of a global trend. Therefore, we assume that the sub-spaces that are close to the global trend are more affected than the ones more distant. Figure~\ref{fig:smoothing} (B) shows how we classify the sub-spaces where dark gray sub-spaces are more influenced by the global trend than the light gray ones that are the neighbors of the dark gray ones. The white sub-spaces are not within the influence area.

As shown in Algorithm~\ref{alg:smoothing}, all sub-spaces are visited in our algorithm to perform individual smoothing operations (line 1, 2). In each visit, the average magnitude is computed with consideration of the neighbor sub-spaces (line 3-12).  
%For each sub-space, the algorithm applies the smoothing operations to the individual sectors independently (line 2).
%For each sector, the algorithm calculates the average magnitude of the same sectors at the neighbor sub-spaces (line 3-12).
Then, our algorithm computes interpolation based on original magnitude of $sp_i$, the average magnitude of neighbor sub-spaces, and the global magnitude based on a local smoothing parameter $\lambda$ and a global smoothing parameter $\tau$ (line 13-18). Finally, the algorithm updates the magnitude of each sector for each sub-space based on the interpolation results. 

Figure~\ref{fig:taxi_glyph} shows the local and the global directional densities trends of taxi movements with our glyph-based visualization in southern Manhattan during in the morning on May 24th, 2013.
Each sub-space has a set of arrows where each arrow indicates the corresponding representative vector.
The orange long arrow represents the global movement in the area.
The yellow lines are actual taxi trajectories.
We will explain details of the data in Section~\ref{sec:case_study}.
The local directional densities and the global movement move toward bottom-left, as the observed time window is in the morning.
Figure~\ref{fig:smoothdemo} shows an example smoothing result where the directional density of the sub-space $A$ changes along the local directional trend. Note that before the smoothing process is performed, the sub-space $B$ did not have the directional density due to data sparseness. After smoothing, it has one accommodating the local directional trend.


%We define the global movement of a sub-space as a unit vector.

%We extract the major movements by clustering the trajectories of the entire space.


\begin{algorithm}[tbh]\footnotesize
\algsetup{linenosize=\footnotesize}
    \SetKwInOut{Input}{Input}
    \SetKwInOut{Output}{Output}
		
    \Input{(1) $\Omega = \left\{sp_1, \cdots, sp_n\right\}$ \\
					 (2) Global vector of $sp_i \enspace G_{sp_i} $ \\
					 (3) A local smoothing parameter $\lambda$ \\%(0 \leq \lambda \leq 1)$ \\
					 (4) A global smoothing parameter $\tau$ \\ %(0 \leq \tau \leq 1)$
					Note: $(0 \leq \lambda + \tau \leq 1)$
					}
    \Output{Smoothed flow of each sub-space}
		
		\tcc{For each sub-space, smooth its directional density}
    \For{\textbf{each} $sp_i \in \Omega$}{
			\tcc{For each sector, adjust its magnitude based on the local and global directional trends}
			\For{$k = 1$ to $S$}{
				\textbf{let} $avgNeighborMag = 0$ \\
				\textbf{let} $numNeighbor = 0$ \\
				\tcc{For each neighbor sub-space of $sp_i$}
				\For{\textbf{each} $sp_j \in N(sp_i)$}{
					$m = M(V_k)$, $V_k \in \mathcal{R}_{sp_j}$ \\
					\If{($m > 0$)}{
						$avgNeighborMag = avgNeighborMag + m$ \\
						$numNeighbor = numNeighbor + 1$
					}
				}
				$avgNeighborMag = avgNeighborMag / numNeighbor$ \\
				$originalMag = M(V_k), V_k \in \mathcal{R}_{sp_i}$ \\
				\tcc{Update the original magnitude}
				\eIf{($D(G_{sp_i})$ is in sector k)}{
					$originalMag = (1 - \lambda - \tau) \times originalMag + \lambda \times avgNeighborMag + \tau \times M(G_{sp_i})$
				}{
					$originalMag = (1 - \lambda) \times originalMag + \lambda \times avgNeighborMag$
				}
			}
		}
		
		%$tolerance\leftarrow t$ \tcp{Initialize temporal tolerance.}
		%
		%\tcp{Loop the user list and calculate the suspicious scores.}
		%\For{each user $u_i$}{
			%\For{each timestamp $t_j$}{
				%\If{$u_i$ posted at least one message in the time range [$t_j$ $\pm$ tolerance]}{
					%$score[u_i]\leftarrow score[u_i]+1$
				%}
			%}
		%}
    \caption{Flow Smoothing}
		\label{alg:smoothing}
\end{algorithm}



%
%\begin{itemize}
  %\item Some grids do not have summarized vectors, due to the data sparseness.
  %\item The vector fields are not smooth[need rephrase here].
%\end{itemize}

%In order to address the two issues, we apply smoothing to the vector field by utilizing both global flow trends and local flow trends.


%
%\textbf{Smoothing based on global trends:}
%
%\textbf{Smoothing based on local trends:} In order to perform smoothing based on local trends, for each grid, we consider the neighbor grids. 
%For example, in the matrix of fixed-size rectangular grids, we utilize the 8 neighbor grids for smoothing. 
%We apply a simple yet effective smoothing operation based on the idea of linear interpolation.
%The details are shown in the following formula.
%
%***FORMULA HERE***
