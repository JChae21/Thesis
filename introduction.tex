%
%  revised  introduction.tex  2011-09-02  Mark Senn  http://engineering.purdue.edu/~mark
%  created  introduction.tex  2002-06-03  Mark Senn  http://engineering.purdue.edu/~mark
%
%  This is the introduction chapter for a simple, example thesis.
%


\chapter{Introduction}

%Background
Since the high global Internet penetration rate and the Web 2.0 era, humans have become a biggest data source.
Humans extremely fast generate a variety of big data using multiple devices, such as personal computers, smartphones and tablets in multiple environments, such as social networks and (micro)blogs.
The data generated by humans is worth understanding, estimating, and predicting their behavior in many areas including marketing, research, and public administration and management.
Also recent advances in technology have enabled people to add location information to social networks called Location-Based Social Networks (LBSNs) where millions of people share their communication and whereabouts
%; where they are, what they are doing and watching, and their thoughts 
not only in their daily lives, but also during abnormal situations, such as crisis events.
Such spatiotemporal data not only provides location-embedded information, but also bring new solutions to a wide range of challenges in analyzing social behaviors and interaction in the physical world.

%Problem Statements
However, the data has challenging issues.
Since the volume of the data exceeds the boundaries of human analytical capabilities and normal computing performance, it is almost impossible to perform a straightforward qualitative analysis of the data.
Also, the contents of the data are usually unstructured and have a high degree of noise.
%; we cannot expect when, where, and who generate the data.
%Thus, it is challenge to find valuable information and patterns from the data.
To address these challenges, researchers have proposed visual analytics that is defined as ``the science of analytical reasoning facilitated by interactive visual interfaces"~\cite{Thomas:VA:2005}.
Currently, many visual analytics techniques integrating approaches from data mining, statistics, and human computer interaction have been proposed to combine the computing power of machines and human analytical capabilities.

In this thesis, we focus on visual analytics techniques of well-known social networks (i.e., Twitter, YouTube, Instagram, Flicker) for decision support in crisis management.
We propose three visual analytics approaches that provide users with scalable and interactive visual spatiotemporal social media data analysis.
In this chapter, the following three sections describe an overview of each visual analytics approach and Section~\ref{sec:statement} provides our thesis statement.


\section{Visual Analytics of Location-based Social Networks for Abnormal Event Detection}
%\section{Spatiotemporal Social Media Visual Analytics for Abnormal Event Detection}

Internet users from all over the world have created a large volume of time-stamped, geo-located data.
Such spatiotemporal data has immense value for increasing situational awareness of local events, 
providing insights for investigations and understanding the extent of incidents, their severity, and consequences, as well as their time-evolving nature.
%However, as data volumes have increased beyond the capabilities of manual evaluation, there is a need for advanced tools to aid in understanding and gleaning investigative insights from the data.
In analyzing social media data, researchers have mainly focused on finding temporal trends according to volume-based importance.
Thus, a relatively small volume of relevant messages for situational awareness are usually buried by a majority of irrelevant data. 
Finding and examining these messages without smart aggregation, automated text analysis and advanced filtering strategies is almost impossible and extracting meaningful information is even more challenging.

In this thesis, we present a visual analytics approach that provides users with scalable and interactive social media data analysis and visualization including the exploration and examination of abnormal topics and events within various social media data sources, such as Twitter, Flickr and YouTube. 
In order to find and understand abnormal events, the analyst can first extract major topics from a set of selected messages and rank them probabilistically using Latent Dirichlet Allocation (LDA)~\cite{Blei:2003:LDA}, which extracts and probabilistically ranks major topics contained in textual parts of the social media data.
%Then abnormality is calculated from the major topical events using seasonal-trend decomposition. 
The ranks of the categorized topics generally provide a volume-based importance,
but this importance does not reflect the abnormality or criticality of the topic. 
In order to obtain a ranking suitable for situational awareness tasks, we discard daily chatter
by employing a Seasonal-Trend Decomposition procedure based on Loess smoothing (STL)~\cite{Cleveland:1990:SAS}.
%We then apply seasonal trend decomposition together with traditional control chart methods to find unusual peaks and outliers within topic time series.
%We apply our method to various social media data and exploit the results in order to improve reliability of our abnormality detection. 
%Our case studies show that situational awareness can be improved by incorporating the anomaly and trend examination techniques into a highly interactive visual analysis process.
Our whole analysis process, including the application of automated tools, is guided and informed by an analyst using a highly interactive visual analytics environment. 
It provides tight integration of semi-automated text-analysis and probabilistic event detection tools together with traditional zooming, filtering and exploration following the Information-Seeking Mantra~\cite{Shneiderman:1996:TEH}.


%\section{Visual Analytics of Social Media for Spacial Decision Support}
\section{Visual Analytics for Public Behavior Analysis in Disaster Events}

%Analysis of public behavior plays an important role in crisis management, disaster response, and evacuation planning. 
For emergency and disaster management, analysis of public behavior, such as how people prepare and respond to disasters, is important.
Unfortunately, collecting relevant data can be costly and finding meaningful information for analysis is challenging. 
As social media has played a pervasive role in the way people think, act, and react to the world, 
%social media is changing the way people communicate not only in their daily lives, but also during abnormal events, such as natural disasters.
in emergency situations, people even seek social confirmation before acting in response to a situation, where they interact with others to confirm information and develop a better informed view of the risk~\cite{national:2013:Public}.
A growing number of LBSNs provides time-stamped, geo-located data that opens new opportunities and solutions to a wide range of challenges.
%The growing dataset of \textit{Location-based Social Network} services with providing time-stamped, geo-located data opens new opportunities and potential solutions.
Such spatiotemporal data has substantial potential to increase situational awareness of local events and improve both planning and investigation. 
However, finding meaningful information from social media is challenging because the large volume of unstructured social media data hinders exploration and examination.
Even though we could extract certain information from the data, it is not always easy to determine whether the analysis result of the extracted information is meaningful and helpful.
%Thus, there is a need for advanced tools to handle such big data and aid in examining the results in order to understand situations and glean investigative insights.
Given the incomplete, complex, context-dependent information, a human in this analysis and decision-making loop is crucial.
Therefore, a visual analytics approach offers great potential through interactive, scalable, and verifiable techniques, helping analysts to extract, isolate, and examine the results interactively.

In this research, we present an interactive visual analytics approach for spatiotemporal microblog data analysis to improve emergency management, disaster preparedness, and evacuation planning. 
We demonstrate the ability to identify spatiotemporal differences in patterns between emergency and normal situations, and analyze spatial relationships among spatial distributions of microblog users, locations of multiple types of infrastructure, and severe weather conditions.
Furthermore, we show how both spatiotemporal microblog and disaster event data can help the analysts to understand and examine emergent situations.


%\section{Trajectory-based Visual Analytics for Anomalous Human Movement}
\section{Visual Analytics of Human Movement Analysis}

Analysis of human movement patterns are important for urban planning~\cite{Zheng:2011:UCW}, traffic forecasting~\cite{Wei:2012:CPR}, and understanding the pandemic spread of diseases~\cite{Eubank:2004:Modelling}.
For crisis and disaster events, movement analysis, such as where people move to/from and how people respond to disasters, is also critical for evacuation management.
%, disaster response, and evacuation planning in crisis management.
%The rapid development and increasing availability of mobile communication and location acquisition technologies allow people to add location data to existing social networks so that people share location-embedded information. 
For human movement analysis, LBSNs have been gaining attention as promising data sources for analyzing human movements.
Particularly, trajectories\textemdash sequences of geo-referenced data nodes of each user\textemdash extracted from such LBSNs provide opportunities and solutions to challenges in human movement analysis~\cite{Andrienko:2009:Analysis, Fuchs:2013:Extracting, Gabrielli:2014:Tweets}.
In addition, semantic context of the data enhances understanding of local events and human movements~\cite{Hochman:2012:Visualizing, Zin:2013:Knowledge}.

Previous studies have mainly focused on finding regular movement patterns using spatial data.
They have demonstrated that human movements are normally influenced by geographic constraints, life patterns, and spatial and temporal events, such as local festivals and holiday seasons~\cite{Andrienko:2011:Movement, Fujisaka:2010:DOU}.
%However, the research may have limitations.
However, during disaster events, since human movement patterns (e.g., volume and direction of movements) are unusual compared to normal situations, a new approach is required to analyze the movements.
Also, analyzing location data alone has shown limitations in achieving situational awareness of local events.
For example, they cannot answer why people move and what situations occur.

To address these challenges, we propose a trajectory-based visual analytics system for anomalous human movement analysis during disasters using multi-type online media.
%We extract trajectories from LBSNs and cluster the trajectories into sets of similar sub-trajectories in order to discover common human movement patterns. 
Our system extracts geo-location information of each data node from LBSNs and generates trajectories using the information.
The generated raw trajectories, however, do not have enough fine-grained spatial positions.
We supplement the sparse positions in the trajectories using route information between each position.
%The complemented trajectories are visualized for further examinations.
We group the individual trajectories into classes of similar sub-trajectories using a trajectory clustering model based on the partition-and-group framework~\cite{Lee:2007:Trajectory}.
This enables users to discover sub-common patterns, rather than finding common patterns as a whole.
%even though there is no common pattern if the basic unit of clustering is the whole trajectory.
%Our system allows users to track and examine change of movement patterns over time.%: past and current.
We also propose a classification model based on historical data for detecting abnormal movements using human expert interaction.
In addition, we integrate multiple visual representations using relevant context extracted from different online media sources, such as Tweet text, shared photos, public webcam videos, and news media to allow users to discover and analyze anomalous human movement patterns; thereby, improving situational awareness in disaster management situations. 

\section{ Thesis Statement}
\label{sec:statement}

%My approaches
This thesis presents the design and development of scalable and interactive visual spatiotemporal social media data analytics techniques for detection and examination of abnormal events, spatial decision support in crisis management, and anomalous human movement analysis.
%for decision support.
%Our visual analytics techniques specifically focus on spatiotemporal social media data for detection and examination of abnormal topics and events and spatial decision support in disaster management and evacuation planning.
%We also propose a trajectory-based visual analytics system for analyzing anomalous human movements and situational awareness improvement using multiple LBSNs and online media (i.e., news media, web camera videos).
The major contributions of this work are the following:

\begin{itemize}
	\item Abnormal topic detection within social media data by combining the STL and the LDA topic model
	%\item Cross validation of abnormal topics among multiple social media sources for improving situational awareness
	\item Design of visual analytics system that enables integration of LBSN data with geo-spatial disaster and infrastructure data for supporting spatial decision-making in crisis management
	%\item Analysis of human mobility responses before, during, and after natural disaster events using interactive visualization
	%\item Discovery and explorer of common human movement patterns from unstructured, massive location-based micro-blog data (i.e., Twitter)
	\item Common human movement pattern discovery from LBSNs using a trajectory clustering model based on the partition-and-group framework
	\item Abnormal human mobility pattern detection and visualization using a trajectory-based anomaly detection model
	\item Development of visual means to improve human movement analysis using semantic context available from multiple online media sources
\end{itemize}

%Future work
As future work, we plan to develop predictive and interactive analytics techniques based on spatiotemporal social media data for integrating visual analytics approaches with automated data analysis models for predicting human movements.
%we plan to research on visual analytics for prediction of human spatial behavior using social media data.
We will design a prediction model using historical trajectory data for forecasting movement patterns and finding next hot spots.
Human movements are affected by multiple factors including times of day (e.g., morning, evening), geographical features (e.g., school, bank), and specific events (e.g., sports game, festival).
Thus, we will investigate and design models based on these context information to enhance the prediction model.
%Also,  multiple context information, such as extracted topics from social media, times of day (e.g., morning vs. evening), and geographical semantic information (e.g., home, school, working places), will be utilized to improve the prediction models.
%We will also advance the prediction models using multiple context information, such as topics, moving time (e.g., morning vs. evening), and geographical semantic information (e.g., home, school, working places).





























