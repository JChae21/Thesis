%
%  revised  front.tex  2011-09-02  Mark Senn  http://engineering.purdue.edu/~mark
%  created  front.tex  2003-06-02  Mark Senn  http://engineering.purdue.edu/~mark
%
%  This is ``front matter'' for the thesis.
%
%  Regarding ``References'' below:
%      KEY    MEANING
%      PU     ``A Manual for the Preparation of Graduate Theses'',
%             The Graduate School, Purdue University, 1996.
%      TCMOS  The Chicago Manual of Style, Edition 14.
%      WNNCD  Webster's Ninth New Collegiate Dictionary.
%
%  Lines marked with "%%" may need to be changed.
%

  % Dedication page is optional.
  % A name and often a message in tribute to a person or cause.
  % References: PU 15, WNNCD 332.
\begin{dedication}
  This is the dedication.
\end{dedication}

  % Acknowledgements page is optional but most theses include
  % a brief statement of apreciation or recognition of special
  % assistance.
  % Reference: PU 16.
\begin{acknowledgments}
  This is the acknowledgments.
\end{acknowledgments}

  % The preface is optional.
  % References: PU 16, TCMOS 1.49, WNNCD 927.
% \begin{preface}
%   This is the preface.
% \end{preface}

  % The Table of Contents is required.
  % The Table of Contents will be automatically created for you
  % using information you supply in
  %     \chapter
  %     \section
  %     \subsection
  %     \subsubsection
  % commands.
  % Reference: PU 16.
\tableofcontents

  % If your thesis has tables, a list of tables is required.
  % The List of Tables will be automatically created for you using
  % information you supply in
  %     \begin{table} ... \end{table}
  % environments.
  % Reference: PU 16.
\listoftables

  % If your thesis has figures, a list of figures is required.
  % The List of Figures will be automatically created for you using
  % information you supply in
  %     \begin{figure} ... \end{figure}
  % environments.
  % Reference: PU 16.
\listoffigures

  % List of Symbols is optional.
  % Reference: PU 17.
% \begin{symbols}
%   $m$& mass\cr
%   $v$& velocity\cr
% \end{symbols}

  % List of Abbreviations is optional.
  % Reference: PU 17.
\begin{abbreviations}
	LBSN& Location-Based Social Network\cr
	LDA& Latent Dirichlet Allocation\cr
	STL& Seasonal-Trend Decomposition procedure based on Loess smoothing\cr
	VF& Vector Field\cr
	LIC& Line Integral Convolution\cr
	ALIC& Animating Line Integral Convolution\cr
	OLIC& Oriented Line Integral Convolution\cr
\end{abbreviations}

  % Nomenclature is optional.
  % Reference: PU 17.
% \begin{nomenclature}
%   Alanine& 2-Aminopropanoic acid\cr
%   Valine& 2-Amino-3-methylbutanoic acid\cr
% \end{nomenclature}

  % Glossary is optional
  % Reference: PU 17.
% \begin{glossary}
%   chick& female, usually young\cr
%   dude& male, usually young\cr
% \end{glossary}

  % Abstract is required.
  % Note that the information for the first paragraph of the output
  % doesn't need to be input here...it is put in automatically from
  % information you supplied earlier using \title, \author, \degree,
  % and \majorprof.
  % Reference: PU 17.
\begin{abstract}
%Background
%Since the high global Internet penetration rate and the Web 2.0 era, humans has been serve the biggest data source. 
%Humans extremely fast generate a variety of big data using multiple devices, such as smartphones and tablets in multiple environments, such as social networks and (micro)blogs.
%Nowadays, humans extremely fast generate a variety of big data using multiple devices, such as personal computers, smartphones and tablets in multiple online environments, such as social networks and (micro)blogs.
%The data generated by humans is significantly worth understanding, estimating, and predicting their behavior in many areas, for example, marketing, research, and public administration and management.
%Nowadays, humans generate a variety of big data.
Recent advances in technology have enabled people to add location information to social networks called Location-Based Social Networks (LBSNs) where people share their communication and whereabouts
%; where they are, what they are doing and watching, and their thoughts 
not only in their daily lives, but also during abnormal situations, such as crisis events.
%Such spatiotemporal data not only provides location-embedded information, but also bring new solutions to a wide range of challenges in analyzing social behaviors in the physical world.	
%Such spatiotemporal data from location-based social networks (LBSNs) has immense value for increasing situational awareness and opens new solutions to a wide range of challenges in analyzing social behaviors and interaction in the physical world.
%Problem Statements
%However, the data has some challenging issues.
However, since the volume of the data exceeds the boundaries of human analytical capabilities, it is almost impossible to perform a straightforward qualitative analysis of the data.
%Also, the contents of the data are usually unstructured and have a high degree of noise.
%; we cannot expect when, where, and who generate the data.
%Thus, it is challenging to find valuable information and patterns from the data.
%My approaches
%This thesis presents visual analytics techniques of LBSNs for decision support.
%We propose a visual analytics approach that provides users with scalable and interactive visual spatiotemporal social media data analysis for detection and examination of abnormal topics and events within various social media data sources.
%We also introduce visual analytics of geo-tagged microblog data that provides a spatial decision support environment in crisis management.
%Finally, we propose a trajectory-based visual analytics system for analyzing anomalous human movements and improving situational awareness using multiple data sources including multi-LBSNs and online media (i.e., news media, web camera videos).
This thesis presents the design and development of scalable and interactive visual spatiotemporal social media data analytics techniques and systems for detection and examination of abnormal events, spatial decision support in crisis management, anomalous human movement analysis, and situational awareness improvement using multi-online media.
%multiple social and traditional online media.
This research couples probabilistic topic models, statistical time series modeling, and trajectory-based clustering models, with interactive visual analytics environments and demonstrates their efficacy.
%Future work
As future work, we plan to research visual analytics for predicting human spatial behavior based on LBSNs.
%using social media data.
We will design a prediction model using historical trajectory data for forecasting movement patterns and finding the next hot spots.
Also,  multiple context information, such as extracted topics from social media, times of day (e.g., morning, evening), and geographical features (e.g., home, school, work place), will be utilized to improve the prediction model.
%We will also advance the prediction models using multiple context information, such as topics, moving time (e.g., morning vs. evening), and geographical semantic information (e.g., home, school, working places).


%Recent advances in technology have enabled social media services to support space-time indexed data, and internet users from all over the world have created a large volume of time-stamped, geo-located data. 
%Such spatiotemporal data has immense value for increasing situational awareness of local events, 
%providing insights for investigations and understanding the extent of incidents, their severity, and consequences, 
%as well as their time-evolving nature.
%In analyzing social media data, researchers have mainly focused on finding temporal trends according to volume-based importance. 
%Hence, a relatively small volume of relevant messages may easily be obscured by a huge data set indicating normal situations. 
%In this paper, we present a visual analytics approach that 
%provides users with scalable and interactive social media data analysis and visualization including the exploration and examination of abnormal topics and events within various social media data sources, such as Twitter, Flickr and YouTube. 
%In order to find and understand abnormal events, the analyst can first extract major topics from a set of selected messages and rank them probabilistically using Latent Dirichlet Allocation. 
%%Then abnormality is calculated from the major topical events using seasonal-trend decomposition. 
%He can then apply seasonal trend decomposition together with traditional control chart methods to find unusual peaks and outliers within topic time series.
%%We apply our method to various social media data and exploit the results in order to improve reliability of our abnormality detection. 
%Our case studies show that situational awareness can be improved by incorporating the anomaly and trend examination techniques into a highly interactive visual analysis process.

%Analysis of public behavior plays an important role in crisis management, disaster response, and evacuation planning. 
%Unfortunately, collecting relevant data can be costly and finding meaningful information for analysis is challenging. 
%A growing number of Location-based Social Network services provides time-stamped, geo-located data that opens new opportunities and solutions to a wide range of challenges.
%%The growing dataset of \textit{Location-based Social Network} services with providing time-stamped, geo-located data opens new opportunities and potential solutions.
%Such spatiotemporal data has substantial potential to increase situational awareness of local events and improve both planning and investigation. 
%However, the large volume of unstructured social media data hinders exploration and examination. 
%To analyze such social media data, our system provides the analysts with an interactive visual spatiotemporal analysis and spatial decision support environment that assists in evacuation planning and disaster management. 
%We demonstrate how to improve investigation by analyzing the extracted public behavior responses from social media before, during and after natural disasters, such as hurricanes and tornadoes.

%Analysis of human movement patterns are important for urban planning, understanding the pandemic spread of diseases, disaster response, and evacuation planning in crisis management. 
%The rapid development and increasing availability of mobile communication and location acquisition technologies allow people to add location data to existing social networks so that people share location-embedded information. 
%For human movement analysis, such location-based social network services have been gaining attention as promising data sources. 
%Researchers have mainly focused on finding daily activity patterns and detecting outliers.
%However, during crisis events, since the movement patterns are irregular, a new approach is required to analyze the movements.
%Also, analyzing location data alone is limited in achieving situational awareness of the events.
%To address these challenges, we propose a trajectory-based visual analytics system for analyzing anomalous human movements during disasters using multi-online media. 
%We extract trajectories from location-based social media and cluster the trajectories into sets of similar sub-trajectories in order to discover common human movement patterns. 
%We also propose a classification model based on historical data for detecting abnormal movements using human expert interaction.
%%For anomaly analysis, we estimate the abnormality of the discovered movement patterns based on historical normal situation patterns. 
%In addition, we integrate multiple visual representations using relevant context extracted from different online media sources. This enhances the human movement analysis by improving situational awareness.
\end{abstract}