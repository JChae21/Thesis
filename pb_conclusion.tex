%\vspace{-0.3cm}

\section{Discussion and Evaluation}
\label{sec:discussion}
%
In this work we found out that the public responses to disaster events in social media streams are different according to the disaster event types.
Hurricane Sandy had a long time duration\textemdash more than one week, and affected a wide range of areas.
Therefore, there were many reactions in the potential damage area before the hurricane impacted the area.
%We could also indicate some hotspots in the area during the event.
However, no or significantly less hotspots were found right after the hurricane passed over the area.
This was because the hurricane severely affected the areas\textemdash communication facility damage and power outages occurred in the area.
Moreover, we found out that unusual post-event situations in the Twitter user distribution continued for a certain time period from a couple of days to more than one week as shown in Figure~\ref{fig:east_coast} and~\ref{fig:graph}.
%~\ref{fig:atlantic_coast}.
The analysts could estimate how long it took for the reconstructions in the areas.
%Hurricane-force winds can extend outward to about from 25 miles to more than 300 miles from its center.

Regarding the tornado case, we intended to find abnormal patterns in the Twitter user distribution before and during the disaster event
but there was no unusual patterns in the area.
In contrast to the hurricane, the tornado generally affected the areas relatively shortly, for example, a few minutes to an hour.
The abrupt natural disaster did not strongly influence the social media stream before and even during the event.
However, as shown in Figure~\ref{fig:tornado}, we were able to find many hotspots within the damaged areas after the tornado passed.
In fact, the tornado damaged some small areas (i.e., a couple of miles wide), in contrast to the wide range of damaged areas for the hurricane case.
This indicated that communication facilities were still available and many people were interested in the disaster, similar to the hurricane.
Thus, our social media analysis could support the analysts to make plans and manage for the emergent situations according to the types of the disasters.
%The number of Tweets are very different between rural and urban areas.

The above cases demonstrate how our system supports spatial decision making through evaluation of varying-density population area to determine changes in behavior, movement, and increase overall situational assessment. This increased spatial activity and behavioral understanding provides rapid situational assessment and provides insight into evolving situational needs to provide appropriate resource allocation and other courses of action (e.g., traffic rerouting, crowd control).

%There is a use case scenario of disaster management support utilizing our visual analytics system.
%Analysts can locate the high density population spots and understand the local situations using our system before or during a disaster event. 
%Then, they can quickly and appropriately respond, such as traffic control to the situations or resource allocation in order to mitigate potential risks.
We requested informal feedback for the usability of our system from users within our universities, and received useful and positive comments and suggestions.
They were interested in the findings of the abnormal situations during the disaster events in Section~\ref{sec:spatial_analysis} and~\ref{sec:temporal_pattern}. They also noted that the use of the infrastructure symbols on the heatmaps improved the legibility of the Twitter user distributions in Figure~\ref{fig:heatmap_manhattan}
and they suggested a visualization for the deviations between multiple heatmaps in order to show the differences clearly, which we plan to develop in the future.
%Furthermore, we are planning to conduct a formal experiment for user evaluation for our system in the future.
%Our system provides a testbed for evaluating the impact of interactive spatiotemporal visual analytics using social media data in disaster management.
%Furthermore, we are planning to conduct a formal user evaluation for 
%the impact of interactive spatiotemporal visual analytics using social media data on disaster management.
%We plan to evaluate: (1) the effect of geospatial visual support to explore and examine public spatial distribution using microblog data; (2) the effect of supplementary spatial information support for spatial decision-making (vs. without any supplementary information); (3) the usability of the glyph-based spatiotemporal visualization; and (4) the performance of the system under different locations, different types of disasters, and different location-based social media data.

%\begin{itemize}
%\item the effect of having geospatial visual support for exploring and examining public spatial distribution using microblog data;
%\item the effect of supplementary spatial information support for spatial decision-making (vs. without any supplementary information);
%\item the usability of the glyph-based spatiotemporal visualization;
%\item and the performance of the system under different locations, different types of disasters, and different location-based social media data.
%\end{itemize}


\section{Summary}
\label{sec:conclusions}
%
In this work we presented a visual analytics system for public behavior analysis and response planning in disaster events using social media data.
We proposed multiple visualizations of spatiotemporal analysis for disaster management and evacuation planning.
For the spatial decision support, we demonstrated an analytical scheme by combining multiple spatial data sources.
Our temporal analysis enables analysts to verify and examine abnormal situations.
Moreover, we demonstrated an integrated visualization that allows spatial and temporal aspects within a single view.
We have still some limitations with these techniques including the potential occlusion issues in the spatiotemporal visualization.
%For future work, we will investigate the flow of public movement before and after disasters and the analysis for recovering from disasters and crises. We also plan to design the glyphs with varied sizes adapting to the zoom level in the spatiotemporal visualization. 
%In addition, 
%%we will conduct a user evaluation of our system in the future.
%we will conduct a user evaluation for the usability and effectiveness of the geospatial visual support,
%and the impact of interactive spatiotemporal visual analytics using social media data on disaster management.
%, the effect of supplementary spatial information support for spatial decision-making (vs. without any supplementary information); (3) the usability of the glyph-based spatiotemporal visualization; and (4) the performance of the system under different locations, different types of disasters, and different location-based social media data.